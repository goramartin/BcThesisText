\chapter{Předpoklady}
\label{requirements}

\section{Labeled-property datový model}
\label{req.propGraph}

Při zpracování této práce uvažujeme grafovou databázi, která pracuje s grafovým modelem \textbf{Labeled-property}\footnote{\url{https://en.wikipedia.org/wiki/Graph_database}}.
V průběhu práce budeme daný model označovat jako \textbf{Property graf}.
V této sekci popíšeme Property graf a jeho vlastnosti z pohledu grafové databáze.
Překvapivě neexistuje standartní definice daného formátu, ačkoliv je značně využíván v moderních systémech.
Proto budeme pracovat s popisem Property grafu systému Neo4j\footnote{\url{https://neo4j.com/developer/graph-database/}} a definicí jazyka PGQL\footnote{\url{https://pgql-lang.org/spec/1.2/\#property-graph-data-model}}.
Na základě těchto popisů určíme klíčové vlastnosti Property grafu, které budeme uvažovat v naši práci.

\subsubsection{Obecný popis Property grafu}
Obecně grafová dazabáze pracující s daným modelem organizuje grafová data do tří částí:

\begin{itemize}
\item První část obsahuje množinu \textbf{vrcholů} (Vertices).
Vrcholy zde představují entity nebo doménové komponenty grafu.
Vrchol tedy reprezentuje objekt, osobu, myšlenku, dílo a podobně.

\item Druhá část obsahuje množinu \textbf{hran} (Edges).
Hrany jsou zde vždy orientované a spojují dva vrcholy.
Hrany zde představují vztahy mezi vrcholy.
Hrana tedy může být například vztah kamarádství dvou lidí (Karel \textit{se kamarádí s} Jirkou.), vztah zaměstnání člověka se zaměstnavatelem (Karel \textit{pracuje pro Google.}) a podobně.
Vrcholy a hrany zde označíme za \textbf{elementy} grafu.
\item Poslední třetí část představuje \textbf{vlastnosti} (Properties) a \textbf{štítky} (Labels), které jsou uloženy v elementech grafu.

\begin{itemize}
\item
Štítek představuje označení určité skupiny nebo chování.
Každému elementu může být přiděleno několik štítků.
Na elementy, kterým byl přidělen stejný štítek, se můžeme dívat jako na skupinu.
Pokud má nějaký element štítek, budeme o štítku hovořit jako o \textbf{typu elementu v Property grafu}.

\item
Vlastnosti definují dvojice (\texttt{název, hodnota}).
První z dvojice \texttt{název} udává název vlastnosti a \texttt{hodnota} udává hodnotu dané vlastnosti.
Hodnota vlastnosti má také vždy definován svůj skalární \textbf{typ} (například číselná hodnota nebo řetězec).
Každému elementu může přináležet několik vlastností.
Avšak, jeden element nemůže mít dvě dvojice sdílející název.
Navíc pokud dva různé elementy sdílejí vlastnost (tj. mají stejný název), pak typy hodnoty vlastnosti musí být rovněž totožné.
\end{itemize}
\end{itemize}

\subsubsection{Omezení Property grafu}
Popsali jsme obecně Property graf.
Daný popis modelu je značně abstraktní, proto si určíme další zpřísňující vlastnosti, které budeme využívat v naši práci.

\begin{enumerate}

\item
Každý element bude mít unikátní identifikátor (\texttt{ID}), abychom dokázali rozlišovat dané elementy.

\item
Každý element (vrchol/hrana) bude mít právě jeden štítek.
Myslíme, že počet štítků není určující pro naši práci, protože primárně ovlivňuje prohledávání grafu a ne části Order by a Group by.

\item
Budeme uvažovat, že štítky vrcholů jsou vždy rozdílné od štítků hran.

\item
Každému typu elementu Property grafu přiřadíme výčet vlastností.
To znamená, že štítek nemusí mít žádnou vlastnost nebo jich má určitý počet.

\item
Štítky mohou sdílet vlastnosti, ale vlastnosti musí mít stejný typ hodnoty.
To znamená, že i štítek hrany se štítkem vrcholu může mít stejnou vlastnost, pokud mají stejný typ hodnoty.

\end{enumerate}

Samotná grafová data nebudeme nijak omezovat.
To znamená, že v grafu mohou být cykly, dva vrcholy mohou být propojeny několika hranami, nemusí existovat žádná hrana ani vrchol, mohou existovat nepropojené vrcholy, hrana může mít totožný počáteční vrchol s koncovým vrcholem a podobně.

\subsubsection{Příklad grafu splňující model}
Nyní uvedeme příklad jednoduchého grafu splňující náš model



\section{PGQL}
\label{req.pgql}

\section{Paralelismus}
\label{req.paralel}
