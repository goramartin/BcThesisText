\chapter{Závěr}
%\addcontentsline{toc}{chapter}{Závěr}

Tady ma byt text

\section{Budoucí výzkum}

Hlavní náplní budoucí práce může být testování daných řešení na grafech s reálnými Properties.
V našem testování jsme sice volili reálne grafy, ale jejich Properties jsme uměle vygenerovali.

Další budoucí výzkum může sledovat obecný problém rozdělení dat při paralelizaci vylepšených řešení. 
Normal přistup má vždy všechna data připravená v paměti a při zpracování je rovnoměrně rozděluje mezi vlákny.
Vlákna tedy mají vždy stejný počet výsledků pro zpracování.
Navíc díky kompletnosti dat lze data optimálněji zpracovávat a použít větší množství obecných algoritmů.
Například při třídění jsme použili základní algoritmu Merge sort, který není možný aplikovat při třídění v průběhu vyhledávání.  
Rozdělení práce vylepšených řešení závísí na počtu vyhledaných výsledků v každém vlákně.
Mohou nastávát případy, kdy jedno vlákno má více výsledků ke zpracování než ostatní. 
Daný problém jsme se v našem řešení prohledávání snažili vyřešit pomocí přidělování malých skupin vrcholů vláknum.
Vlákno po prohledání daných vrcholů zažádalo o další.
Nicméně, dané řešení nemůže zaručit stoprocentně rovnoměrné rozdělení práce.
Bylo by vhodné prozkoumat, jak daná situace ovlivňuje naše řešení.

U Order by řešení jsme viděli značné zrychlení při třídění pomocí Properties v paralelizovaných řešeních.
Bylo by vhodné prozkoumat možnosti vytvoření globálních statistik pro každou Property a podrobněji zjistit možnosti rozdělení rozsahů přihrádek.
Samotné rozdělení přihrádek jsme pro řetězce zpracovali pouze s předpokladem, že se jedná o ASCII znaky.
V budoucí práci je možné zkoumat rozdělování i pro složitější znakové sady.

V paralelních Group by řešeních by bylo vhodné prozkoumat podrobněji skalabilitu daných řešení pro rozličné počty vláken.
Pokud možno, také možnosti jiných paralelních map.
 