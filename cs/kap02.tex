

\chapter{Analýza}

V této kapitole se pokusíme analyzovat možná řešení výstavby dotazovacího enginu a jeho úpravy, tak abychom splnili námi kladené požadavky společně se zadáním práce.
Analýza bude probíhat v několika krocích.
Začneme obecným návrhem dotazovacího enginu a projdeme hlavní koncepty pro implementaci.
V druhém kroku zvážíme kroky vykonávání dotazů a postup výběru řešení částí Order by a Group by, které se budou vykonávat po dokončení vyhledávání dotazu.
V třetím kroku provedeme analýzu úprav pro agregaci v průběhu vyhledávání. 
Součástí této části bude analýza algoritmů Order by a Group by pro dané úpravy. 

\section{Obecný pohled na engine}

V naší představě je dotazovací engine určen pro práci nad grafem, který je celý obsažen v paměti, včetně vlastností elementů grafu.
Graf bude načten v definovém formátu a následně na něm budou vykonávány dotazy.
V momentě načtění graf bude pouze statický, tedy nebude docházet k žádným změnám.
Nad grafem se pak vykoná uživatelsky definovaný dotaz.
Dané omezení jsme zvolili, protože hlavním cílem je testovat pouze části Group by a Order by.
Vytvořit reálnou grafovou databázi by zabralo netriviální časové období.

Při obecném pohledu na engine jsme lokalizovali hlavní bloky; výstavby, které musíme uvážit.
Jsou to: reprezentace grafu, parsování uživatelského dotazu, výrazy (expressions) a dotaz/vykonání dotazu.
Graf nám bude simulovat grafovou databázi. 
Samotně pak určuje formát objektů, nad kterými je vykonán uživatelský dotaz.
Při parsování se načítá uživatelký dotaz do interní reprezentace.
Expressions slouží k výpočtu hodnot z uživatelsky zadaných výrazů.
Například v části order by x.PropOne, musíme vědět, jak reprezentovat výraz x.PropOne a získat jeho hodnotu. 
Na základě interní reprezentace se musí vytvořit struktury dotazu a definovat exekuční plán. 
Z obecných úkonů částí vidíme, že se nejedná o stand-alone části.
Vytváří se nám závislosti, které budeme muset uvážit.

\section{Reprezentace grafu}

Musíme uvážit, jak reprezentovat graf.
Z části \ref{requirements} jsou hlavními faktory námi zvolená podmnožina jazyku PGQL a že se jedná o Property graf.
Pro případy nejednoznačnosti označíme \verb+elType+ jako typ elementu v Property grafu a \verb+propType+ jako typ Property.

\subsection{Elementy grafu a jejich typ}

Musíme zvažovat reprezentaci elementů grafu a jejich \verb+elType+.
V našem případě jsou elementy pouze vrcholy a orientované hrany.
\verb+elType+ definuje seznam Properties na elementu. 
Properties jsou také typované.
Vrchol a hrana musí mít rozdílný \verb+elType+, ale samotné Properties se mohou opakovat pro oba druhy elementů.
Každá hodnota Property musí být přístupná skrze daný element:

\begin{itemize}

\item Pokud držíme element grafu, musíme být schopni jej rozlišit od ostatních elementů.

\item Pokud držíme element grafu, musíme být schopni přistoupit k hodnotám jeho Properties.

\end{itemize}

V naší představě je řešení následovné.
Elementy budou třídy.
Každý element grafu bude potomkem jednoho abstraktního předka a potomci si budou definovat svá specifika.
Potomek bude vrchol a hrana.
Předek si bude pamatovat unikátní \verb+ID+, abychom elementy dokázali rozlišit. 
Předek navíc bude znát svůj \verb+elType+. 
Bude se jednat o ukazatel na třídu.
Daná třída by reprezentovala pouze jeden \verb+elType+ a bude společná všem elementům majících daný \verb+elType+.
V třídě by byl obsažen seznam \verb+IDs+ elementů daného typu, jejich pořadí (např: dle vkládání do seznamu) a Properties v podobě polí s hodnotami.
Property musí být přístupná skrze mapu/slovník, protože může nastat situace, kdy daná Property na elementu neexistuje. 
Pro náš případ nebudeme uvažovat situaci, kdy Property pro nějaký element nemá definovanou hodnotu.
Properties tedy budou přístupné pomocí unikátního identifikátoru pro celý graf.
Hodnoty Properties každého elementu by ležely na pozicích dle pořadí \verb+IDs+.
Nyní, pokud držíme element grafu, můžeme přistoupit k hodnotě Property skrze tabulku pomocí jeho \verb+ID+.    
Samotný přistup pak může být realizován například generickou funkcí. 

\subsection{Struktury obsahující elementy}

Nyní musíme analyzovat jaké struktury by byly idální pro uchovávání elementů grafu.
Musíme brát v potaz, že propojení mezi vrcholy pomocí hran přímo ovlivňuje vyhledávání v části Match.
V průběhu vyhledávání v určitý moment vždy držíme odkaz na nějaký element grafu.
Na základě daného elementu musíme provést akci:

\begin{itemize}

\item Pokud držíme vrchol, musíme být schopni přistoupit k jeho hranám. Hranám z/do něj. Daný přístup by měl být co nejrychlejší a neměl by obsahovat žádné iterace. V průběhu hledání se z vrcholu musí projít skrze všechny jeho hrany. Ideálně by měly být hrany přístupné skrze index.

\item Pokud držíme hranu, musíme být schopni přistoupit ke koncovému vrcholu. V průběhu hledání vždy vlastníme vrchol než přistoupíme k jeho hraně a následně k jejímu koncovému vrcholu. Tímto můžeme vyloučit nutnost, aby hrana znala informaci o svém původu.

\item Pokud držíme element grafu, chceme být schopni přistoupit k jeho sousedním elementů v obsajující struktuře za předpokladu, že víme, jestli se jedná o hranu nebo vrchol. 

\end{itemize}

K vyřešení daných problému v naší představě bychom použili tři pole.
Pole vrcholů, pole out hran a in hran. 
Zde by bylo vhodné vytvořit nové potomky obecné hrany: out hrana a in hrana.
Hrany by si pamatovali svůj koncový vrchol.
Pro in hranu by to byl vrchol odkud vychází, aby bylo možné v moment držení vrcholu projít skrze ni na vrchol další.
Každé pole tedy bude mít unikátní typ, který nám pomůže rozlišit k jaké situaci má dojít v průběhu prohledávání.
Abstraktní předek všech elementů by si měl nově pamatovat i svou pozici v daných polích pro rychlý přístup k jeho sousedům.

Zbývá vyřešit vztah hran a vrcholů.
Řešení, které bychom chtěli zvolit, je mít hrany v polích seskupeny podle: vrcholů odkud vycházejí (pole out hran), vrcholů kam směřují (pole in hran).
Vrchol by si pak pamatoval rozsah svých hran v příslušných polích. 
Chceme-li procházet hrany vrcholu, stačí procházet pole out/in hran pomocí rozsahů uložených v daném vrcholu.
Tedy čtyř indexů.
Skrze indexy můžeme pak pole libovolně iterovat.

Uvažovali jsme nad různými alternativami. 
Mít jeden typ hrany obsahující všechny nutné informace.
Řešení je paměťově přijatelnější, ale nastává problém s přístupem k in hranám vrcholu.
Řešením by mohlo být vytvořit pole in/out hran pro každý vrchol. 
Daný přístup nám případá výrazně náročnější z hlediska paměti, protože musíme vytvářet pole pro každý vrchol zvlášť. 

\subsection{Vstupní grafová data}

Vstupní soubory musí obsahovat informace nutné pro Property graf.
Budeme očekávat dva druhy souborů.
Soubory schémat typů elementů a jejich Properties.
Datové soubory pak budou obsahovat konkrétní data elementů.

Protože každý element grafu má svůj \verb+elType+, budeme mít na vstupu dva soubory schémat pro hrany a vrcholy.
Schéma bude obsahovat informace o všech \verb+elType+ a \verb+propType+ vyskytujících se v grafu.
Pro \verb+elType+ je důležitý název a výčet Properties.
Properties pak musí nést svůj název a \verb+propType+.
Vidíme, že se jedná jen o výčet \verb+(name/value)+ dvojic (např. \verb+(PropertyOne, integer)+).
V tomto případě se nám jeví nejvhodnější zvolit pro reprezentaci schémat formát JSON.
\verb+elType+ bude reprezentován JSON objektem. 
Bude obsahovat položku \verb+Kind+, jejíž hodnota udává jméno \verb+elType+.
Za ní bude následovat výčet Properties.
Properties budou reprezentovány dvojicí \verb+(propName/propType)+.
Záznamy pak budou obsaženy v JSON poli:
\begin{code}
Soubor schéma vrcholů:
[    { "Kind": "BasicNode" }, 
     { "Kind": "BasicNodeTwo", "PropertyOne": "integer" } ]

Soubor schéma hran:"
[    { "Kind": "BasicEdge" }, 
     { "Kind": "BasicEdgeTwo", "PropertyOne": "integer" } ]
\end{code}
Jako určující \verb+propTypes+ pro náš případ enginu bychom chtěli zvolit dva druhy.
První by byla číselná hodnota značená integer (32-bit integer).
Další by představoval řetězec značený string.
Práce s řetězci je obecný problém a existuje mnoho znakových sad, proto bychom chtěli zvolit vstupní řetězce pouze se znaky ASCII.  
Dané dva druhy představují základní typy použitých v komerčních sférách.

Samotná data budou obsažena opět ve dvou separatních souborech pro hrany a vrcholy.
Chtěli bychom reprezentovat konkrétní data pomocí jednoduchého .csv souboru.
Každý řádek reprezentuje jednu hranu/vrchol.
V první řadě řádek musí obsahovat unikatní \verb+ID+ elementu a jeho \verb+elType+. 
Za \verb+elType+ následuje seznam hodnot Properties v pořadí určených schématem.
Pro hrany existuje na řádku navíc záznam \verb+ID+ vrcholů, které spojuje.
Oddělovače mezi daty jsou implementační detail.
Pro naše účely se jedná o dostačující formát a poskytuje nám jednoduché možnosti parsování.
Pokud by docházelo v budoucnu k rozšířením, například více slovné Property nebo XML Property, musí dojít k úpravě daných formátů.

Pro výše zmíněné schéma by datové soubory mohly vypadat následovně:
\begin{code}
Soubor hran:
ID elType fromID toID Properties // bez této hlavičky
50 BasicEdge 0 0 
51 BasicEdgeTwo 0 1 44
...
Soubor vrcholů:
ID elType Properties // bez této hlavičky
0 BasicNode
1 BasicNodeTwo 42
...
\end{code}

\section{Parsování uživatelského dotazu}

Máme rozmyšlenou reprezentaci grafu.
Nyní musíme analyzovat způsob získání informací z uživatelem zadaného dotazu.
Uživatelský dotaz se pohybuje v rozsahu definovaném v sekci PGQL \ref{req.pgql}.
Nicméně, je zde nutné přemýšlet i nad možnými rozšířeními.
Proto se budeme snažit držet základních principů object oriented programming a volit vhodné návrhové vzory.

K načtení uživatelského dotazu se nám jeví jako nejvhodnější způsob použít techniky známé z překladačů programovacích jazyků.
Budeme vycházet ze základních principů knihy o překladačích \citep{dragoonBook}.
V prvním kroku dojde k lexikální analýze uživatelsky zadaného řetězce.
Dojde k vytvoření tokenů.
V druhém kroku dojde k syntaktické a semantické analýze tokenů.
Metodou top-down parsing \citep[str. 217]{dragoonBook} se vytvoří stromová struktura reprezentující daný dotaz.
Poslední krok provede vytvoření tříd reprezentující dotaz pomocí iterace stromové struktury.
Iterace a sběr dat ze stromové struktury budou implementovány návrhovým vzorem Visitor \citep[str. 331]{patterns}.
V naši představě bychom chtěli vygenerovat stromovou strukturu pro každou hlavní část dotazu (Match, Select, Order by a Group by).
Nyní bychom mohli sestavit Visitor pro každou část separatně a vyřadit tak nutnost jednoho globálnáho Visitoru.
Dané postupy nám umožní jednoduše pracovat s naší podmnožinou jazyka PGQL.


\subsection{Match a proměnné}

Každá hlavní část dotazu po sesbírání informací pomocí Visitoru vygeneruje určité struktury.
Pro Match se přímočaře naskytuje reprezentovat posloupnosti vrcholů a hran pomocí polí.
Pole bude obsahovat třídy.
Třída si musí pamatovat jakou proměnou reprezentuje, \verb+elType+ pokud je definován a jde-li o hranu (in/out/any) nebo vrchol.
Jedná se o všechny nutné informace, které můžeme následně využít k vytvoření vzoru prohledávání grafu.
Všimout si musíme faktu, že Match část definuje proměnné ve zbytku dotazu.
Vyvstává problém nutnosti vytvořit mapu/slovník přistupných proměnných pro zbytek dotazu.
Proměnným pak můžeme přiřadit \verb+ID+.

\subsection{Select, Order/Group by}

Ostatní části Group by, Order by a Select obsahují výrazy proměnných (např: order by x), přístup k Properties proměnných (např: select x.PropOne) nebo volání agregačních funkcí (\verb+min+, \verb+max+, \verb+avg+, \verb+sum+ a \verb+count+).
Várazy se však musí evaluovat za běhu programu.
Dalším problémem je, že dané výrazy mají různorodé návratové hodnoty.
Výraz x (\verb+ID+ vrcholu) lze chápat jako integer.
Váraz x.PropOne má návratovou hodnotu dle \verb+propType+, který je definovám ve vstupním schématu.
Agregační funkce \verb+min+, \verb+max+ mají návratovou hodnotu definovanou na základě jejich vstupních argumentů.
Například \verb+min(x.StringProp)+ očekává, že návratová hodnota bude řetězec.
Funkce \verb+sum+ a \verb+count+ by měli ideálně vracet formát, který by předešel přetečení.
U \verb+avg+ se očekává hodnota s desetinnou čárkou.  
V budoucnu však může dojít k rozšířením a vyvstanou složitější výrazy, např. infixová notace x.PropOne + y.PropOne.
Problém nám usnadňuje fakt, že Properties nesoucí stejné jméno mají stejný \verb+propType+ (sekce \ref{req.pgql}).
Pokud ne, je nutné určit vhodnou návratovou hodnotu.
Navíc musíme brát v potaz, že daný výraz se nemusí vyhodnotit, například absence Property na vrcholu.
Proto jsme byli nuceni vymyslet systém výrazů (expressions).

\subsection{Expressions}

Systém vytváření a vyhodnocování výrazů efektivně za běhu je obecně složitý problém.
Omezíme se pouze na případy: přístup k proměnné, přístup k hodnotě Property proměnné a agregační funkce (\verb+min+, \verb+max+, \verb+avg+, \verb+sum+ a \verb+count+).

Základní myšlenka je reprezentovat výraz pomocí stromové struktury. 
Každý vrchol stromové struktury bude reprezentovat určitou akci.
Vrcholy budou výše vypsané výrazy. 
Na struktuře bude existovat metoda pro vyhodnocení.
Její návratová hodnota bude dvojice úspěch vyhodnocení + vypočtená hodnota. 
Dané struktury musí být read-only, protože se budou využívat v paralelním prostřední.
Metody by se mohli libovolně dodávat při nutnosti použití nových struktur. 
Následuje ukázka možného kódu pro přístup k \verb+ID+ proměnné v jazyce C\#:
\begin{code}
// Base classes
abstract class Expression { }
abstract class ExpressionReturnValue<T>: Expression {
  public abstract bool TryEvaluate(Element[] elms, out T retVal); 
}

abstract class VariableAccess<T>: ExpressionReturnValue<T> {
     readonly int accessedVariableID; 
}

class VariableIDAccess: VariableAccess<int> {
  public override bool TryEvaluate(Element[] elms, out int retVal) {
     returnValue = elms[accessedVariableID].ID;
     return true; }}
\end{code}
Třída \verb+VariableAccess+ nám poskytuje abstrakci pro přístup k proměnné.
Položka \verb+accessedVariableID+ určuje k jaké proměnné se má přistoupit.
Zde předpokládáme, že pole \verb+Element[]+ obsahuje proměnné přesně v pořadí, jak se vyskytly v části Match.
Tedy pokud je roven Match (x) -> (y), tak jeden výsledek hledání by bylo pole obsahující dva elementy x a y.
Jedná se pouze o ilustrační příklad. 
Případ přístupu k Property by mohl vypadat následovně:
\begin{code}
class VariablePropertyAccess<T>: VariableAccess<T> {
  reaonly int accessedPropertyID; 
  public override bool TryEvaluate(Element[] elms, out T retVal) {
    return elms[accessedVariableID].
               GetPropertyValue<T>(accessedPropertyID, out retVal);
  }
}
\end{code}
Zde dojde k volání metody na elementu grafu, který přistoupí k třidě reprezentující jeho \verb+elType+.
Třída pak na základě existence Property vrátí hodnotu nebo neuspěje.
Timto dokážeme vyřešit základní definované problémy.

Zbává uvažovat, jakým způsobem reprezentovat agregační funkce.
Agregační funkce představují několik problémů.
Funkce je vypočtená pouze pro skupiny. 
Skupiny jsou vytvářeny v části Group by.
Jejich návratové hodnoty jsou finální pouze po dokončení Group by. 
Vstupem funkcí je expression.
Argument, dle kterého se aktualizuje agregovaná hodnota, je nutný znát pouze v době vykonání Group by.
Dle naši představy je ideální vytvořit dva separátní koncepty.
První koncept bude zahrnovat výpočet hodnot argumentu společně s logikou agregační funkce.
Koncept bude reprezentován třídou, která vlastní stromovou strukturu dle předchozího příkladu.
Zároveň bude obsahovat logiku počítané funkce.
Například logika funkce \verb+min+ je porovnat dvě hodnoty a vybrat menší.
Na vstupu dané funkce pak bude úložiště hodnoty dané skupiny.
Všechny počítané agregační funkce zadané uživatelem označíme pomocí \verb+ID+.
Druhý koncept představuje nový potomek třídy expression.
Daný potomek si pamatuje \verb+ID+ přistupované agregační funkce a na vstupu očekává strukturu reprezentující skupinu.
K hodnotě počítané agregace přistoupíme pomocí její \verb+ID+.
Následuje ukázka prvního konceptu:
\begin{code}
abstract class Aggregation { }
abstract class Aggregation<T>: Aggregation {
  public ExpressionReturnValue<T> expr; // Argument of the agg. func.
  public abstract void Apply(ValueStorage storage, Element[] elms);
}

public class Sum<T>: Aggregation<T>{
  public override void Apply(ValueStorage storage, Element[] elms) {
    if (expr.TryEvaluate(elms, out T retVal)) {
      storage.value += retVal;
    }
  }
}
\end{code}
Zde vidíme položku \verb+expr+, která reprezentuje vstupní expression agregační funkce.
Funkce \verb+Apply+ je logikou funkce. 
Vidíme funkci \verb+Sum+. 
Logikou je přičtení vypočítané hodnoty do poskytnutého úložište, pokud dojde k úspěšné evaluaci výrazu.

Následuje ukázka druhého konceptu:
\begin{code}
class GroupAggValueAccess<T>: ExpressionReturnValue<T> {
  reaonly int accessedAggregationFuncID; 
  public override bool TryEvaluate(Group group, out T retVal) {
    retVal = group.GetAggValue<T>(accessedAggregationFuncID);
    return true;
  }
}
\end{code}
\verb+accessedAggregationFuncID+ je \verb+ID+ vypočítané agregační funkce.
Hodnota funkce se vrací pomocí \verb+GetAggValue<T>+.
Tímto jsme vyřešeli problémy parsování a reprezentace výrazů pro náš engine.
Nyní přistoupíme k problémům vykonávání a reprezentace dotazu.

\section{Reprezentace dotazu}

\section{Group by}

\section{Order by}

\section{Návrh vylepšení}