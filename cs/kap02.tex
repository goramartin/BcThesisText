

\chapter{Analýza implementace}

V této kapitole se pokusíme analyzovat možná řešení výstavby dotazovacího enginu a jeho úpravy, tak abychom splnili námi kladené požadavky společně se zadáním práce.
Analýza bude probíhat v několika krocích.
Začneme obecným návrhem dotazovacího enginu a projdeme hlavní koncepty pro implementaci.
V druhém kroku zvážíme kroky vykonávání dotazů a postup výběru řešení částí Order by a Group by, které se budou vykonávat po dokončení vyhledávání dotazu.
V třetím kroku provedeme analýzu úprav pro agregaci v průběhu vyhledávání. 
Součástí této části bude analýza algoritmů Order by a Group by pro dané úpravy. 

\section{Obecný pohled na engine}

V naší představě je dotazovací engine určen pro práci nad grafem, který je celý obsažen v paměti, včetně vlastností elementů grafu.
Graf bude načten v definovém formátu a následně na něm budou vykonávány dotazy. 
Při obecném pohledu na engine jsme lokalizovali hlavní bloky výstavby.
Jsou to tři bloky: reprezentace grafu, parser a reprezentace dotazu.

Reprezentace grafu určuje pohled na načtený graf v paměti a určuje formát objektů, nad kterými bude vykonáván dotaz.
Parser načítá uživatelký dotaz do interní reprezentace. 
Na základě reprezentace se vytváří struktury dotazu a definuje se exekuční plán. 
Z obecných úkonů částí vidíme, že se nejedná o stand-alone části.
Vytváří se nám závislosti, které budeme muset uvážit.

\section{Reprezentace grafu}

Musíme uvážit, jak reprezentovat graf.
Z části \ref{requirements} je hlavní faktor podmnožina jazyku PGQL pro Property grafy.

\subsection{Elementy grafu a jejich typ}

Musíme zvažovat reprezentaci elementů grafu a jejich typu.
V našem případě jsou elementy pouze vrcholy a orientované hrany.
Typ definuje seznam Properties na elementu. 
Properties jsou také typované.
Každá hodnota Property musí být přístupná skrze daný element:

\begin{itemize}

\item Pokud držíme element grafu, musíme být schopni jej rozlišit od ostatních elementů.

\item Pokud držíme element grafu, musíme být schopni přistoupit k jeho vlastnostem.

\end{itemize}

\clearpage

V naší představě je řešení následovné.
Každý element grafu bude potomkem jednoho abstraktního předka a potomci si budou definovat svá specifika.
Potomek bude vrchol a hrana.
Předek si bude pamatovat unikátní IDs v celém grafu, abychom elementy dokázali rozlišit. 
Předek navíc bude znát svůj typ. 
Ideálně ukazatel na tabulku.
Tabulka by reprezentovala typ a byla by společná všem elementům majících stejný typ.
V tabulce by byl obsažen pouze seznam IDs, jejich pořadí a Properties v podobě polí s hodnotami.
Properties by byly přístupné pomocí unikátního identifikátoru pro celý graf.
Hodnoty Properties každého elementu by ležely na pozicích dle pořadí IDs.
Nyní, pokud držíme element grafu, můžeme přistoupit k hodnotě Property skrze tabulku pomocí jeho ID.    

\subsection{Struktury obsahující elementy}

Nyní musíme analyzovat, jaké struktury by byly idální pro uchovávání elementů grafu.
Musíme brát v potaz, že propojení mezi vrcholy pomocí hran přímo ovlivňuje vyhledávání v části Match.
V průběhu vyhledávání v určitý moment vždy vlastníme nějaký element grafu.
Na základě daného elementu musíme provést akci:

\begin{itemize}

\item Pokud držíme vrchol, musíme být schopni přistoupit k jeho hranám. Hranám z/do něj. Daný přístup by měl být co nejrychlejší a neměl by obsahovat žádné iterace. V průběhu hledání se z vrcholu musí projít skrze všechny jeho hrany. Ideálně by měly být hrany přístupné skrze index.

\item Pokud držíme hranu, musíme být schopni přistoupit ke koncovému vrcholu. V průběhu hledání vždy vlastníme vrchol než přistoupíme k jeho hraně a následně k jejímu koncovému vrcholu. Tímto můžeme vyloučit nutnost, aby hrana znala informaci o svém původu.

\item Pokud držíme element, chceme být schopni přistoupit k jeho sousedním elementů v obsajující struktuře. Například budeme chtít prohledávát graf jen z určeného množství vrcholů bez vytváření pomocných struktur.

\end{itemize}

K vyřešení daných problému v naší představě bychom použili tři pole.
Pole vrcholů, pole out hran a in hran. 
Zde by bylo vhodné rozšířit potomky hran na dva nové typy.
Hrany by si pamatovali svůj koncový vrchol.
Pro in hranu by to byl vrchol odkud vychází, aby bylo možné v moment držení vrcholu projít skrze ni na vrchol další.
Každé pole tedy bude mít unikátní typ, který nám pomůže rozlišit k jaké situaci má dojít v průběhu prohledávání.
Abstraktní předek všech elementů by si měl nově pamatovat i svou pozici v daných polích pro rychlý přístup k jeho sousedům.
Zbývá vyřešit vztah hran a vrcholů.
Řešení, které bychom chtěli zvolit, je mít hrany v polích seskupeny podle: vrcholů odkud vycházejí (pole out hran), vrcholů kam směřují (pole in hran).
Vrchol by si pak pamatoval rozsah svých hran v příslušných polích. 
Chceme-li procházet hrany vycházejích z vrcholu, stačí držet pole out hran a rozsah náležící vrcholu. Tedy dva indexy.
Skrze indexy můžeme pak libovolně iterovat.

Uvažovali jsme nad různými alternativami. 
Mít jeden typ hrany obsahující všechny nutné informace.
Řešení je paměťově přijatelnější, ale nastává problém s přístupem k in hranám vrcholu.
Řešením by mohlo být vytvořit pole in/out hran pro každý vrchol. 
Daný přístup nám případá výrazně náročnější z hlediska paměti, protože musíme vytvářet pole pro každý vrchol zvlášť. 

\subsection{Vstupní grafová data}

Vstupní soubory musí obsahovat informace nutné pro Property graf.
Budeme očekávat dva druhy souborů.
Soubory schémat typů elementů a jejich Properties.
Datové soubory pak budou obsahovat konkrétní data elementů.

Protože každý element má svůj typ, budeme mít na vstupu dva soubory schémat pro hrany a vrcholy.
Schéma bude obsahovat informace o všech typech vyskytujících se v grafu.
Pro typ je důležitý název a výčet Properties.
Properties pak musí nést svůj název a typ.
Vidíme, že se jedná jen o výčet (key, value) dvojic (např. (PropOne, integer)).
V tomto případě se nám jeví nejvhodnější zvolit formát JSON.
Typ bude reprezentovám JSON třídou. 
Bude obsahovat položku Kind, která udává jméno typu.
Za ní bude následovat výčet Properties.
Properties budou reprezentovány přesně jako dvojice výše.
Záznamy pak budou obsaženy v JSON poli:
\begin{code}
Soubor schéma vrcholů:
[    { "Kind": "BasicNode" }, 
{ "Kind": "BasicNodeTwo", "PropOne": "integer" } ]

Soubor schéma hran:
[    { "Kind": "BasicEdge" }, 
{ "Kind": "BasicEdgeTwo", "PropOne": "integer" } ]
\end{code}

Samotná data budou obsažena opět ve dvou separatních souborech.
Každý řádek by reprezentoval jednu hranu/vrchol.
V první řadě řádek musí obsahovat unikatní ID elementu a jeho typ. 
Za typem by následoval seznam hodnot Properties v pořadí určených schématem.
Pro hrany by existoval na řádku navíc záznam ID vrcholů, které spojuje.
Oddělovače mezi daty jsou implementační detail.
Pro výše zmíněné schéma by datové soubory mohly vypadat následovně:
\begin{code}
Soubor hran:
ID TYPE FROMID TOID PROPERTIES // bez této hlavičky
50 BasicEdge 0 0 
51 BasicEdgeTwo 0 1 44
...
Soubor vrcholů:
ID TYPE PROPERTIES // bez této hlavičky
0 BasicNode
1 BasicNodeTwo 42
...
\end{code}

\subsection{Načítání grafových dat}




\section{Parser}

\section{Expressions a Aggregates}

\section{Reprezentace dotazu}

\section{Group by}

\section{Order by}

\section{Návrh vylepšení}