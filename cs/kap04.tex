\chapter{Experiment}

Aby bylo možné porovnat stávající řešení s nově navrženým řešením na poli rychlosti zpracovávání dotazů, podrobili jsme zmíněná řešení experimentu. 
Vykonaný experiment proběhne na reálných grafech různé velikosti s uměle vygenerovanými vlastnostmi naležící vrcholům. 
Nad danými grafy provedeme vybrané množství dotazů, které nám umožní sledovat a porovnat chování řešení v různých situacích.
Experiment bude zakončen diskuzí nad výsledky. 

\section{Příprava dat}

Pro náš experiment jsme použili tři orientované grafy z databáze SNAP\footnote{\citet{snapnets}}.

\begin{table}[!htb]
\centering
\begin{tabular}{lrr}
\toprule
\mc{} & \mc{\textbf{\#Vrcholů}} & \mc{\textbf{\#Hran}} \\
\midrule
Amazon0601     & 403394 & 3387388 \\
WebBerkStan & 685230   & 7600595 \\
As-Skitter    & 1696415   & 11095298 \\
\bottomrule
\end{tabular}

\caption{Vybrané grafy pro experiment}
\label{tab.grafBase}
\end{table}

\begin{itemize}

\item \textbf{Amazon0601:} Jedná se o graf vytvořený procházením webových stránek Amazonu na základě featury „Customers Who Bought This Item Also Bought“ ze dne 1.6.2003. V grafu existuje hrana z $i$ do $j$, pokud je produkt $i$ často zakoupen s produktem $j$.

\item \textbf{WebBerkStan:} Graf popisuje odkazy webových stránek domén \url{https://www.stanford.edu/} a \url{https://www.berkeley.edu/}. Vrcholem je webová stránka a hrana představuje hypertextový odkaz mezi stránkami.

\item \textbf{As-Skitter:} Topologický graf internetu z roku 2005 vytvořený programem \verb+traceroutes+. Ačkoliv je uvedeno, že daný graf je neorientovaný, vnitřní hlavička souboru uvádí opak, proto jsme se daný graf rozhodli přesto využít.

\end{itemize}

Samotné grafy obsahují pouze seznam hran. Abychom mohli dané grafy využít, bylo nutné je transformovat a vygenerovat k nim Properties na vrcholech. 
Při příkladu transformace budeme vycházet z následující ukázky hlavičky (graf Amazon0601):

\begin{code}
# Directed graph (each unordered pair of nodes is saved once): 
    Amazon0601.txt: 
# Amazon product co-purchaisng network from June 01 2003
# Nodes: 403394 Edges: 3387388
# FromNodeId	ToNodeId
0	1
0	2
0	3
0	4
\end{code}

\subsection{Transformace grafových dat}

Výstupem transformace budou soubory popisující schéma vrcholů/hran NodeTypes.txt/EdgeTypes.txt a datové soubory vrcholů/hran Nodes.txt/Edges.txt.
V našem případě graf bude obsahovat pouze jeden typ hrany a jeden typ vrcholu. Dané omezení ovlivňuje pouze vyhledávání vzoru, které není určující pro náš experiment. 

Ukázka zvoleného schématu pro Nodes.txt/Edges.txt:
\begin{code}
Soubor EdgeTypes.txt:
[
{ "Kind": "BasicEdge" }
]

Soubor NodeTypes.txt:
[ 
{ "Kind": "BasicNode" }
]

\end{code}

Generování souborů Edges.txt/Nodes.txt provádí program, který je obsahem přílohy zdrojových kódů \ref{prilohy.kod} v souboru GrapDataBuilder.cs.
Výstupní soubor Edges.txt bude obsahovat hrany v rostoucím pořadí dle položky \verb+FromNodeId+ z originálního souboru s přidělenými \verb+IDs+ od hodnoty \verb+ID+ posledního vrcholu v souboru Nodes.txt.
Samotný soubor Nodes.txt obsahuje setřizené vrcholy podle \verb+ID+ v rostoucím pořadí. Je nutné zmínit, že setřízení dat podle \verb+ID+ není nežádoucí, jelikož nezaručuje nic o seskupení vrcholů v daném grafu.
Pro připomenutí zmíníme, že prvni sloupeček v datových souborech Edges.txt a Nodes.txt odpovídá unikátnímu \verb+ID+ v rámci celého grafu.

Následuje ukázka výstupních souborů transformace pro graf Amazon0601:  
\begin{code}
Soubor Edges.txt:
403395 BasicEdge 0 1
403396 BasicEdge 0 2
...

Soubor Nodes.txt:
0 BasicNode 
1 BasicNode 
...
\end{code}

\subsection{Generování Properties vrcholů}

Posledním krokem přípravy dat pro experiment je vygenerování Properties vrcholů.
Jsme si vědomi, že nejideálnější způsob testování je graf s reálnými daty, nicméně dané omezení jsme se rozhodli aplikovat kvůli problematickému hledání vhodných dat, které nevyžadují netriviální transformaci do vhodného vstupního formátu.
Proto pro každý vrchol náhodně vygenerujeme hodnoty tří Properties. 
\clearpage
\begin{table}[!htb]
\centering
\begin{tabular}{lll}
\toprule
\mc{\textbf{Property}} & \mc{\textbf{Type}}  & \mc{\textbf{Popis}}\\
\midrule
PropOne     & integer &  \verb+Int32+ s rozsahem $[0, 100000]$ \\
PropTwo & integer   & \verb+Int32+ s rozsahem $[$\verb+Int32.MinValue+$,$ \verb+Int32.MaxValue+$]$ \\
PropThree    & string &  délka $[2, 8]$ ASCII znaků s rozsahem $[33, 126]$ \\
\bottomrule
\end{tabular}

\caption{Generované Properties vrcholů}
\label{tab.grafProps}
\end{table}

\begin{itemize}

\item \textbf{PropTwo} hodnoty jsou rovněž generováný střídavě kladně a záporně, aby nastal rovnoměrný počet záporných a kladných hodnot.

\item \textbf{PropThree} hodnoty jsou pouze ASCII znaky z rozsahu $[33, 126]$. Dané omezení výplývá z vlastností dotazovacího enginu, aby bylo možné bez obtíží načíst datový soubor.

\end{itemize}

Na základě tabulky generovaných Properties \ref{tab.grafProps} následuje ukázka upraveného souboru schématu pro vrcholy:
\begin{code}
Soubor NodeTypes.txt:
[
{ 
"Kind": "BasicNode",
"PropOne": "integer",
"PropTwo": "integer",
"PropThree": "string" 
}
]
\end{code}

Výsledné hodnoty Properties do souborů Edges.txt/Nodes.txt jsou vygenerovány pomocí programu, který používá generátor náhodných čísel. Program je obsažen v příloze zdrojových kódů \ref{prilohy.kod} v souboru PropertyGenerator.cs.
Pro každý graf bylo použité jiné \verb+Seed+ pro inicializaci náhodného generátoru. Samotná \verb+Seeds+ byla vygenerována rovněž náhodně.

\begin{table}[!htb]
\centering
\begin{tabular}{lr}
\toprule
\mc{} & \mc{\textbf{Seed}} \\
\midrule
Amazon0601     & 429185 \\
WebBerkStan &  20022 \\
As-Skitter    & 82 \\
\bottomrule
\end{tabular}

\caption{Inicializační hodnoty náhodného generátoru pro PropertyGenerator.cs}
\label{tab.seeds}
\end{table}

Program generuje hodnoty definované ve statické položce \verb+propGenerators+ a zachovává jejich pořadí ve výsledném datovém souboru.
Aby nedocházelo k omylům při opakování experimentů, uvádíme útržek kódu použité inicializace položky dle tabulky generovaných vlastností \ref{tab.grafProps} pro všechny tři grafy:
\clearpage

\begin{code}
    static PropGenerator[] propGenerators = new PropGenerator[]
    {
        new Int32Generator(0, 100_000, false),
        new Int32Generator(true),
        new StringASCIIGenerator(2, 8, 33, 126)
    };
\end{code}


Timto jsme dokončili poslední nutný krok k vygenerování platných vstupních dat pro dotazovací engine. Použité grafy k transformaci a výsledné datové soubory jsou obsahem přílohy grafů pro experiment \ref{prilohy.grafy}

\section{Výběr dotazů}

Dotazy použité při experimentu dělíme do tří kategorií a to Match, Order by a Group by.
Pro připomenutí zmíníme, že proměnné použité v jiných částech než Match způsobují ukládání daných proměnných do tabulky.
Přidělené zkratky dotazům budou uváděný ve výsledcích experimentu namísto celých dotazů. 

\subsection{Dotazy Match}

Každý dotaz provádí vyhledáváním vzoru v grafu.
Níže zmíněné dotazy nám při experimentu pomohou oddělit čas agregací od času stráveném vyhledáváním vzoru.

\begin{table}[!htb]
\centering
\begin{tabular}{ll}
\toprule
\mc{\textbf{Zkratka}} & \mc{\textbf{Dotaz}} \\
\midrule
M\_Q1 & select count(*) match (x) -> (y) -> (z); \\
M\_Q2 & select x match (x) -> (y) -> (z); \\
M\_Q3 & select x, y match (x) -> (y) -> (z); \\
M\_Q4 & select x, y, z match (x) -> (y) -> (z); \\
\bottomrule
\end{tabular}

\caption{Dotazy Match}
\label{tab.dotazM}
\end{table}

\begin{itemize}

\item \textbf{M\_Q1} testuje pouze dobu strávenou vyhledáváním vzoru.

\item \textbf{M\_Q2} testuje vyhledávání společně s ukládáním proměnné x do tabulky výsledků.

\item \textbf{M\_Q3} testuje vyhledávání společně s ukládáním proměnné x a y do tabulky výsledků.

\item \textbf{M\_Q4} testuje vyhledávání společně s ukládáním proměnné x, y a z do tabulky výsledků.
\end{itemize}

\clearpage

\subsection{Dotazy Order by}

\begin{table}[!htb]
\centering
\begin{tabular}{ll}
\toprule
\mc{\textbf{Zkratka}} & \mc{\textbf{Dotaz}} \\
\midrule
O\_Q1 & select y match (x) -> (y) -> (z) order by y; \\
O\_Q2 & select y, x match (x) -> (y) -> (z) order by y, x;\\
O\_Q3 & select x.PropTwo match (x) -> (y) -> (z) order by x.PropTwo;\\
O\_Q4 & select x.PropThree match (x) -> (y) -> (z) order by x.PropThree \\
\bottomrule
\end{tabular}

\caption{Dotazy Order by}
\label{tab.dotazO}
\end{table}

\begin{itemize}

\item \textbf{O\_Q1} testuje třídění podle \verb+ID+ vrcholů y. 

\item \textbf{O\_Q2} přidává do kontextu \textbf{O\_Q1} overhead za porovnávání a ukládání další proměnné.

\item \textbf{O\_Q3} testuje třídění náhodně vygenerovaných hodnot Int32 (viz \ref{tab.grafProps}).

\item \textbf{O\_Q4} testuje třídění náhodně vygenerovaných řetězců (viz \ref{tab.grafProps}).

\end{itemize}


\subsection{Dotazy Group by}

\begin{table}[!htb]
\scriptsize
\centering
\begin{tabular}{ll}
\toprule
\mc{\textbf{Zkratka}} & \mc{\textbf{Dotaz}} \\
\midrule
G\_Q1 & select min(y.PropOne), avg(y.PropOne) match (x) -> (y) -> (z);\\
G\_Q2 & select min(y.PropOne), avg(y.PropOne) match (x) -> (y) -> (z) group by y;\\
G\_Q3 & select min(y.PropOne), avg(y.PropOne) match (x) -> (y) -> (z) group by y, x;\\
G\_Q4 & select min(x.PropOne), avg(x.PropOne) match (x) -> (y) -> (z) group by x.PropTwo;\\
G\_Q5 & select min(x.PropOne), avg(x.PropOne) match (x) -> (y) -> (z) group by x;\\
G\_Q6 & select min(x.PropOne), avg(x.PropOne) match (x) -> (y) -> (z) group by x, y;\\
G\_Q7 & select min(x.PropOne), avg(x.PropOne) match (x) -> (y) -> (z) group by x.PropOne;\\
\bottomrule
\end{tabular}

\caption{Dotazy Group by}
\label{tab.dotazG}
\end{table}

Pro volbu agregačních funkcí jsme zvolili \verb+min(y.PropOne)+ a \verb+avg(y.PropOne)+, protože představují netriviální práci narozdíl od funkcí \verb+sum+/\verb+count+.
V případě \verb+min(y.PropOne)+ v paralelních řešeních dochází k mechanismu \verb+CompareExchange+ a u \verb+avg(y.PropOne)+ dojde ke dvou atomickým přičtením.

\begin{itemize}

\item \textbf{G\_Q1} testuje single group Group by. Vše je agregováno pouze do jedné skupiny. 

\item \textbf{G\_Q2} testuje vytváření skupin podle \verb+ID+ vrcholů y

\item \textbf{G\_Q3} přidává k \textbf{G\_Q2} overhead za ukládání a zpracovávání (hash + compare) další proměnné.

\item \textbf{G\_Q4} testuje vytváření skupin náhodně vygenerovaných hodnot Int32 (viz \ref{tab.grafProps}). 

\item \textbf{G\_Q7} testuje situaci, kdy dojde k rozprostření několika stejných hodnot v grafu. Dodává větší šanci, že nějaké vlákno během paralelního zpracování dostane stejnou skupinu jako vlákno jiné.

\end{itemize}

\section{Metodika}

Pro provedení experimentu jsme připravili jednoduchý benchmark, který je součástí příloh zdrojových kódů \ref{prilohy.kod}.
Paralelizování řešení jsme otestovali při zatížení všech dostupných jader procesoru (argument \verb+ThreadCount = 8+).
Při spuštění programu dojde k navýšení priority procesu, aby docházelo k méně častému vykonávání ostatních procesů na pozadí během testování. 
Pro \verb+ThreadCount = 1+ navíc dochází k navýšení priority hlavního vlákna. 
To není možné u paralelního testování, protože vlákna běží v nativním \verb+ThreadPool+, který neumožňuje navyšování priority vláken.

Následuje ukázka hlavní smyčky benchmarku:

\begin{code}
...
 WarmUp(...);
...
double[] times = new double[repetitions];
for (int i = 0; i < repetitions; i++)
{
    CleanGC();
    
    var q = Query.Create(..., false); 

    timer.Restart();

    q.Compute();

    timer.Stop();
    times[i] = timer.ElapsedMilliseconds;

    ...
}
\end{code}
Hlavní smyčka benchmarku se skládá z 5-ti opakování warm up fáze následovanou 15-ti opakováními měřené části. 

V konstruktoru \verb+Query.Create(..., false)+ argument \verb+false+ způsobuje, že vykonávaný dotaz neprovede \verb+select+ část dotazu, která není cílem testování.

Před měřením dochází vždy k úklidu haldy. 
\begin{code}
static void CleanGC()
{
    GC.Collect();
    GC.WaitForPendingFinalizers();
    GC.Collect();
}
\end{code}
K měření uplynulé doby  jsme použili nativní třídu C\# \verb+Stopwatch+, protože náš hardware a operační systém podporuje high-resolution performance counter.
Pro interpretaci výsledků jsme zvolili medián naměřených hodnot, který je doprovázen minimem a maximem.

\subsection{Argumenty pro jednotlivé grafy}

Pro měření argumenty \verb+FixedArraySize+ a \verb+VerticesPerThread+ jsme volili následovně:

\begin{table}[!htb]
\centering
\begin{tabular}{lrr}
\toprule
\mc{} & \mc{\textbf{FixedArraySize}} & \mc{\textbf{VerticesPerThread}} \\
\midrule
Amazon0601 &  4194304 & 512 \\
WebBerkStan & 4194304 & 512 \\
As-Skitter & 8388608 & 1024\\
\bottomrule
\end{tabular}

\caption{Výber argumentů konstruktoru dotazu pro grafy}
\label{tab.args}
\end{table}

Dané argumenty se nám nejvíce osvědčili v průběhu vývoje dotazovacího enginu. Vyhledávání vzoru na nich docilovalo nejrychlejších výsledků.

\subsection{Hardwarová specifikace}

Všechny testy proběhly na notebooku Lenovo ThinkPad E14 Gen. 2 verze 20T6000MCK s operačním systémem Windows 10 x64.
\begin{itemize} 
\item 8 jádrový procesor AMD Ryzen 7 4700U (2GHz, TB 4.1GHz)

\item 24GB RAM DDR4 s 3200 MHz
\end{itemize}

Každému testování předcházel studený reboot systému a odpojení od internetu. 
V průběhu testování neběžel žádný klientský proces kromě benchmarku a nativních systémových procesů. 
Rovněž, použitý notebook byl napájen po celou dobu testování.

\section{Výsledky a diskuze}
