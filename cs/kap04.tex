\chapter{Experiment}

Aby bylo možné porovnat stávající řešení s nově navrženým řešením na poli rychlosti zpracovávání dotazů, podrobili jsme
jsme zmíněná řešení experimentu. Vykonaný experiment proběhne na reálných grafech různé velikosti s uměle vygenerovanými
vlastnostmi naležící vrcholům. Nad danými grafy provedeme vybrané množství dotazů, které nám umožní sledovat a porovnat chování řešení v různých situacích. 

\section{Příprava dat}

Pro náš experiment jsme použili tři orientované grafy z databáze SNAP\footnote{\citet{snapnets}}.

\begin{table}[!htb]
\centering
\begin{tabular}{lrr}
\toprule
\mc{} & \mc{\textbf{\#Vrcholů}} & \mc{\textbf{\#Hran}} \\
\midrule
Amazon0601     & 403394 & 3387388 \\
WebBerkStan & 685230   & 7600595 \\
As-Skitter    & 1696415   & 11095298 \\
\bottomrule
\end{tabular}

\caption{Grafy použité k experimentu}
\label{tab.grafBase}
\end{table}

\begin{itemize}

\item \textbf{Amazon0601:} Jedná se o graf vytvořený procházením webových stránek Amazonu na základě featury „Customers Who Bought This Item Also Bought“ ze dne 1.6.2003. V grafu existuje hrana z $i$ do $j$, pokud je produkt $i$ často zakoupen s produktem $j$.

\item \textbf{WebBerkStan:} Graf popisuje odkazy webových stránek domén \url{https://www.stanford.edu/} a \url{https://www.berkeley.edu/}. Vrcholem je webová stránka a hrana představuje hypertextový odkaz mezi stránkami.

\item \textbf{As-Skitter:} Topologický graf internetu vytvořený programem \verb+traceroute+ z roku 2005. Ačkoliv je uvedeno, že daný graf je neorientovaný, vnitřní hlavička souborů uvádí opak, proto jsme se daný graf rozhodli přesto využít.

\end{itemize}

Samotné grafy obsahují pouze seznam hran. Následuje ukázka souboru grafu Amazon0601:

\begin{code}
# Directed graph (each unordered pair of nodes is saved once): 
    Amazon0601.txt: 
# Amazon product co-purchaisng network from June 01 2003
# Nodes: 403394 Edges: 3387388
# FromNodeId	ToNodeId
0	1
0	2
0	3
0	4
0	5
0	6
0	7
\end{code}

\subsection{Transformace grafových dat}

Abychom mohli dané grafy použít, museli jsme je přetransformovat do platného formátu vstupních souborů dotazovacího enginu. Výstupem transformace budou soubory popisující schéma vrcholů/hran NodeTypes.txt/EdgeTypes.txt a samotné datové soubory vrcholů/hran Nodes.txt/Edges.txt.
V našem případě graf bude obsahovat pouze jeden typ hrany a jeden typ vrcholu. Dané omození ovlivňuje pouze vyhledávání vzoru, které není přínosné pro náš experiment, protože snižuje počet nalezených výsledků. 

Cílem transformace je získat formát datových souborů Nodes.txt/Edges.txt odpovídající schématu:
\begin{code}
Soubor EdgeTypes.txt:
[
{
"Kind": "BasicEdge"
}
]

Soubor NodeTypes.txt:
[
{
"Kind": "BasicNode"
}
]

\end{code}

Program provádějící transformaci je obsahem přílohy zdrojových kódů \ref{prilohy.kod} v souboru GrapDataBuilder.cs, který vygeneruje datové soubory Nodes.txt a Edges.txt dle výše vypsaného schématu.
Pro připomenutí zmíníme, že prvni sloupeček v datových souborech Edges.txt a Nodes.txt odpovídá unikátnímu \verb+ID+ v rámci celého grafu.
Výstupní soubor Edges.txt bude obsahovat hrany setřízené v rostoucím pořadí dle položky \verb+FromNodeId+ s přidělenými \verb+IDs+ od hodnoty \verb+ID+ posledního vrcholu v souboru Nodes.txt.
Samotný soubor Nodes.txt obsahuje setřizené vrcholy podle \verb+ID+ v rostoucím pořadí. Je nutné zmínit, že setřízení dat podle \verb+ID+ není nežádoucí, jelikož nezaručuje nic o seskupení vrcholů v daném grafu.

Následuje ukázka výstupních souborů transformace pro graf Amazon0601:  
\begin{code}
Soubor Edges.txt:
403395 BasicEdge 0 1
403396 BasicEdge 0 2
403397 BasicEdge 0 3
403398 BasicEdge 0 4
...

Soubor Nodes.txt:
0 BasicNode 
1 BasicNode 
2 BasicNode 
3 BasicNode
...
\end{code}

\subsection{Generování Properties vrcholů}

Posledním krokem přípravy dat pro experiment je vygenerovat Properties vrcholů.
Jsme si vědomi, že nejideálnější způsob testování je graf s reálnými daty, nicméně dané omezení jsme se rozhodli aplikovat kvůli problematickému hledání vhodných dat s triviální transformací do vhodného vstupního formátu.
Proto pro každý vrchol náhodně vygenerujeme hodnoty tří Properties. 

\begin{table}[!htb]
\centering
\begin{tabular}{lll}
\toprule
\mc{\textbf{Property}} & \mc{\textbf{Type}}  & \mc{\textbf{Popis}}\\
\midrule
PropOne     & integer &  \verb+Int32+ s rozsahem $[0, 100000]$ \\
PropTwo & integer   & \verb+Int32+ s rozsahem $[$\verb+Int32.MinValue+$,$ \verb+Int32.MaxValue+$]$ \\
PropThree    & string &  délka $[2, 8]$ ASCII znaků s rozsahem $[33, 126]$ \\
\bottomrule
\end{tabular}

\caption{Generované Properties vrcholů}
\label{tab.grafProps}
\end{table}

\begin{itemize}

\item \textbf{PropTwo} hodnoty jsou rovněž generováný střídavě kladně a záporně, aby nastal rovnoměrný počet záporných a kladných hodnot.

\item \textbf{PropThree} hodnoty jsou pouze ASCII znaky z rozsahu $[33, 126]$. Dané omezení výplývá z vlastností dotazovacího enginu, aby bylo možné bez obtíží načíst datový soubor.

\end{itemize}

Na základě tabulky generovaných Properties \ref{tab.grafProps} následuje ukázka upraveného souboru schématu pro vrcholy:
\begin{code}
Soubor NodeTypes.txt:
[
{
"Kind": "BasicNode",
"PropOne": "integer",
"PropTwo": "integer",
"PropThree": "string"
}
]
\end{code}

Výsledné hodnoty Properties do souborů Edges.txt/Nodes.txt jsou vygenerovány pomocí programu, který používá generátor náhodných čísel. Program je obsažen v příloze zdrojových kódů \ref{prilohy.kod} v souboru PropertyGenerator.cs.
Pro každý graf bylo použité jiné \verb+Seed+ pro inicializaci náhodného generátoru v daném programu. Samotná \verb+Seeds+ byla vygenerována rovněž náhodně.

\begin{table}[!htb]
\centering
\begin{tabular}{lr}
\toprule
\mc{} & \mc{\textbf{Seed}} \\
\midrule
Amazon0601     & 429185 \\
WebBerkStan &  20022 \\
As-Skitter    & 82 \\
\bottomrule
\end{tabular}

\caption{Inicializační hodnoty náhodného generátoru pro PropertyGenerator.cs}
\label{tab.seeds}
\end{table}

Aby nedocházelo k omylům při opakování experimentů, uvádíme útržek kódu použitého nastavení programu PropertyGenerator.cs dle tabulky generovaných vlastností \ref{tab.grafProps}:
\begin{code}
    static PropGenerator[] propGenerators = new PropGenerator[]
    {
        new Int32Generator(0, 100_000, false),
        new Int32Generator(true),
        new StringASCIIGenerator(2, 8, 33, 126)
    };
\end{code}

Timto jsme dokončili poslední nutný krok k vygenerování platných vstupních dat pro dotazovací engine. Použité grafy k transformaci a výsledné datové soubory jsou obsahem přílohy grafů pro experiment \ref{prilohy.grafy}

\section{Výběr testovaných dotazů}

V této sekci provedeme výběr a popis dotazů, které budou objektem testování.

\section{Metodika}

C
\subsection{HW}

\section{Výsledky a diskuze}
