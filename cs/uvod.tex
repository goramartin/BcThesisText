\chapter*{Úvod}
\addcontentsline{toc}{chapter}{Úvod}

Tady ma byt text.

\section*{Cíl práce}

\begin{enumerate}

\item
Navrhnout dotazovací engine, který bude sloužit jako výchozí prostředí pro naši práci.
Obsahuje dvě části:
\begin{enumerate}
\item První část obsahuje jednu \textbf{statickou grafovou databázi}, která pracuje s \textbf{Labeled-property} modelem (sekce \ref{req.propGraph}) grafových dat.
Celá grafová databáze bude obsažena v hlavní paměti.

\item Druhá část obsahuje algoritmy a struktury pro zpracování dotazu \textbf{podmnožiny jazyka PGQL} (sekce \ref{req.pgql}) nad danou grafovou databází:
    \begin{enumerate}
    \item Podmnožina jazyka PGQL bude obsahovat pouze hlavní částí dotazu: \textbf{Select}, \textbf{Match}, \textbf{Group by} a \textbf{Order by}.
    \textbf{Nebude možno} zadat Group by a Order by společně.
    \item Všechna data v průběhu zpracování dotazu budou obsažena v hlavní paměti.
    \item Zpracování částí Group by a Order by bude provedeno \textbf{po dokončení prohledávání grafu} v části Match.
    To znamená, že nejdříve dojde k nalezení všech výsledků prohledávání a teprve pak dojde k seskupení nebo setřídění daných výsledků.
    \item Obecně engine bude schopen vykonat dotaz jednovláknově.
    Části \textbf{Match}, \textbf{Group by} a \textbf{Order by} bude schopen vykonat i paralelně.
    \end{enumerate}
\end{enumerate}

\item
Upravit druhou část výše definovaného dotazovacího enginu po vzoru streamovacích sýstémů tak, aby vykonávání částí Group by a Order by probíhalo v průběhu prohledávání grafu časti Match:
\begin{enumerate}

\item
V moment nalezení jednoho výsledku prohledávání grafu v části Match bude daný výsledek zpracován.
Zpracováním zde rozumíme zatřídění do již setříděné posloupnosti výsledků nebo přiřazením výsledku do patřičné skupiny.

\item 
Engine po úpravě musí být schopen zpracovávat dotazy původním řešením i upraveným řešením.

\item
Upravená část bude pracovat se stejnou podmnožinou jazyka PGQL jako původní neupravená část.

\item
Všechna data v průběhu zpracování budou obsažena v hlavní paměti.

\item
Obecně upravená část bude schopna vykonat dotaz jednovláknově.
Části \textbf{Match}, \textbf{Group by} a \textbf{Order by} bude schopna vykonat i paralelně.
\end{enumerate}

\item
Vykonat experiment, který testuje původní zpracování (odrážka 1.b) vůči upravené části (odrážka 2.) v rychlosti vykonání dotazů:
\begin{enumerate}
\item Experiment bude proveden na několika reálných grafech s reálnými nebo uměle vygenerovanými vlastnostmi.
\item V experimentu otestujeme všechna navrhnutá řešení.
\item Každé řešení bude otestováno na několika vybraných dotazech.
\item Experiment bude zakončen prezentací výsledků a diskuzí.
\end{enumerate}
\end{enumerate}