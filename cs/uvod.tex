\chapter*{Úvod}
\addcontentsline{toc}{chapter}{Úvod}

\section*{Grafové databáze}

Grafové databáze zažívají v dnešní době rozkvět, například kvůli nutnosti analyzovat data na sociálních sítích.
Pro grafové databáze existují dva hlavní modely dat: \textbf{RDF} \citep{rdf} a \textbf{Labeled-property} (sekce \ref{req.propGraph}).
RDF je převážně používán k popisu informací na internetu. 
Labeled-property model se používá k obecnému popisu grafových dat.
Grafové databáze poskytují dvě základní možnosti analýzy dat:
\begin{itemize}
\item
První možnost je \textbf{spouštění algoritmů} nad uloženými grafovými daty.
Samotná skupina používaných algoritmů obsahuje od jednoduchých algoritmů jako hledání nejkratší cesty až po komplexní algoritmy jako Community detection \citep[str. 115]{graphAlg}.
Program, který poskytuje tuto možnost analýzy dat, budeme označovat jako \textbf{analytický engine}.
\item
Druhá možnost je \textbf{vykonávání dotazů} pomocí dotazovacího jazyka.
Dotazy jsou zde podobné SQL dotazům.
Pro RDF model to je například jazyk SPARQL \citep{sparql}.
Pro Labeled-property to je PGQL \citep{pgql} nebo openCypher\footnote{\url{https://www.opencypher.org/} [dostupnost ověřena k datu 8.5.2021]}.
Dotazy obsahují základní části se stejnou logikou z SQL jako Select, Order by, Group by, Having a podobně.
Hlavní rozdíl těchto dotazovacích jazyků od SQL je ten, že specifikují výběr dat pomocí sekvence vrcholů a hran.
Pro PGQL a openCypher to je část dotazu Match a pro SPARQL to je část dotazu Where.
Tyto sekvence tvoří podgraf (vzor), který bude vyhledáván v grafu.
V moment nalezení všech podgrafů jsou na výsledcích aplikovány další části jako Group by nebo Order by.
Tomuto se říká vyhledávání vzoru, nebo-li pattern matching.
Program, který poskytuje tuto možnost analýzy dat, budeme označovat jako \textbf{dotazovací engine}.
\end{itemize}

\section*{Proudové systémy}

Na druhé straně množství dat ke zpracování poslední dobou značně roste a nastává problém při jejich zpracování, protože se všechny nevejdou do paměti nebo dochází k jejich velmi častým změnám.  
Z tohoto důvodu začaly vznikat takzvané \textbf{proudové systémy} \citep{streaming}.
Proudové systémy obecně pracují s daty, která jsou potencionálně nekonečná a nevlezou se do paměti.
V takovém případě jsou data reprezentovány formou tekoucího „\textbf{proudu}“ (stream).
Obecně uvnitř proudu mohou být jakákoliv data, například prováděné akce, události změn nebo transakce.
Programy pracující s proudem dat sledují jen jeho určitý bod a zpracovávají data v moment kdy protékají daným bodem.
Výsledně program vidí jen útržky všech dat.
Zpracování takových dat pak probíhá po částech.

Existují dva hlavní modely zpracování \citep{graphstreaming}.
První model udržuje kondenzovaný stav při zpracování dat a nazývá se \textbf{synopsis}.
Dobrým příkladem může být počítání sumy hodnot prvků v proudu.
V takovém případě si program udržuje pouze sumu výsledků.
Data v proudu pak nepotřebuje uchovávát.
Druhý model si udržuje okno určitého počtu posledních prvků z proudu a nazývá se \textbf{windowing}.
Prvky v okně tvoří skupinu, která se v určitý moment zpracuje najednou.
Proudové systémy se uchytili v distribuovaných systémech, například Apache Flink\footnote{\url{https://ci.apache.org/projects/flink/flink-docs-release-1.13} [dostupnost ověřena k datu 8.5.2021]}.

\section*{Proudové systémy s grafovými daty}

Nově se objevují proudové systémy pracující s grafovými daty.
V proudu jsou pak obsaženy hrany nebo vrcholy. 
Nad daným proudem vzniká nutnost analýzy grafových dat jako u grafových databází\footnote{Přehledovou práci základních algoritmů a principů zpracování je možno nalézt v článku „Graph Stream Algorithms: A Survey“ \citep{graphstreamalgorithms}.}.
Mezi proudové systémy s grafovými daty patří například rozšíření pro Apache Flink Gelly\footnote{\url{https://ci.apache.org/projects/flink/flink-docs-release-1.13/docs/libs/gelly/overview/} [dostupnost ověřena k datu 8.5.2021]}.

Proudové systémy s grafovými daty mají výhodu při \textbf{agregaci} prvků vůči grafovým databázím, protože při použití modelu \textbf{synopsis} jim stačí uchovávat pouze agregované prvky. 
Agregací zde rozumíme akt seskupování prvků podle určitých klíčů (obecně část Group by v SQL, PGQL nebo SPARQL).
Součástí agregace je i výpočet statických funkcí pro vzniklé skupiny.
Tyto funkce označujeme jako agregační funkce.
Mezi nejčastější agregační funkce patří funkce:
\begin{itemize} 
\item \texttt{min(...)}, která vrací minimum z hodnot ve skupině..
\item \texttt{max(...)}, která vrací maximum z hodnot ve skupině..
\item \texttt{count(...)}, která vrací počet prvků ve skupině.
\item \texttt{avg(...)}, která vrací aritmetický průměr z hodnot ve skupině..
\item \texttt{sum(...)}, která vrací sumu z hodnot ve skupině.
\end{itemize}
Problémem je, že proudové systémy nedovolují vykonávat dotazy obsahující vyhledávání vzoru (část Match pro PGQL nebo Where pro SPARQL), protože grafová data čteme po částech a tak je vyhledávání vzoru prakticky nemožné.
Jediným možným způsobem je vytvořit přehledovou strukturu grafu \citep{graphsummary}, na které vyhledávání proběhne.
Avšak tím se ztratí všechny výhody proudových systémů.

\section*{Spojení vyhledávání vzoru s proudovými systémy}

tady bude co vlastne chceme delat na zaklade predchozi sekce.
max jedna stranka a to je vsechno z uvodu.

\clearpage
\section*{Cíl práce}

Hlavním cílem práce je určit zda úprava částí Group by a Order by dotazovacího enginu po vzoru proudových systémů dociluje rychlejšího zpracování dotazů než zpracování, které by dané části vykonávalo po získání všech výsledků prohledávání grafu.
Kroky k dosažení cíle jsou následovné:
\begin{enumerate}
\item
Navrhnout dotazovací engine, který bude sloužit jako výchozí prostředí pro naši práci.
Obsahuje dvě části:
\begin{enumerate}
\item První část obsahuje jednu \textbf{statickou grafovou databázi}, která pracuje s \textbf{Labeled-property} modelem (sekce \ref{req.propGraph}) grafových dat.
Celá grafová databáze bude obsažena v hlavní paměti.

\item Druhá část obsahuje algoritmy a struktury pro zpracování dotazu \textbf{podmnožiny jazyka PGQL} (sekce \ref{req.pgql}) nad danou grafovou databází:
    \begin{enumerate}
    \item Podmnožina jazyka PGQL bude obsahovat pouze hlavní částí dotazu: \textbf{Select}, \textbf{Match}, \textbf{Group by} a \textbf{Order by}.
    \textbf{Nebude možno} zadat Group by a Order by společně.
    \item Všechna data v průběhu zpracování dotazu budou obsažena v hlavní paměti.
    \item Zpracování částí Group by a Order by bude provedeno \textbf{po dokončení prohledávání grafu} v části Match.
    To znamená, že nejdříve dojde k nalezení všech výsledků prohledávání a teprve pak dojde k seskupení nebo setřídění daných výsledků.
    \item Obecně engine bude schopen vykonat dotaz jednovláknově.
    Části \textbf{Match}, \textbf{Group by} a \textbf{Order by} bude schopen vykonat i paralelně.
    \end{enumerate}
\end{enumerate}

\item
Upravit druhou část výše definovaného dotazovacího enginu po vzoru proudových sýstémů tak, aby vykonávání částí Group by a Order by probíhalo v průběhu prohledávání grafu časti Match:
\begin{enumerate}

\item
V moment nalezení jednoho výsledku prohledávání grafu v části Match bude daný výsledek zpracován.
Zpracováním zde rozumíme zatřídění do již setříděné posloupnosti výsledků nebo přiřazením výsledku do patřičné skupiny.

\item 
Engine po úpravě musí být schopen zpracovávat dotazy původním řešením i upraveným řešením.

\item
Upravená část bude pracovat se stejnou podmnožinou jazyka PGQL jako původní neupravená část.

\item
Všechna data v průběhu zpracování budou obsažena v hlavní paměti.

\item
Obecně upravená část bude schopna vykonat dotaz jednovláknově.
Části \textbf{Match}, \textbf{Group by} a \textbf{Order by} bude schopna vykonat i paralelně.
\end{enumerate}

\item
Vykonat experiment, který testuje původní zpracování (odrážka 1.b) vůči upravené části (odrážka 2.) v rychlosti vykonání dotazů.
Cílem experimentu je určit, zda úprava dotazovacího enginu po vzoru proudových systémů způsobí zrychlení vykonávání dotazu:
\begin{enumerate}
\item Experiment bude proveden na několika reálných grafech s reálnými nebo uměle vygenerovanými vlastnostmi.
\item V experimentu otestujeme všechna navrhnutá řešení.
\item Každé řešení bude otestováno na několika vybraných dotazech.
\item Experiment bude zakončen prezentací výsledků a diskuzí.
\end{enumerate}
\end{enumerate}